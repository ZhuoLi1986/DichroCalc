\documentclass[12pt, a4paper]{article}

% #####################################################################################################################
%
% Title:           Documentation for the Perl library GetParameters.pm
%
% Mail address:    webmaster.-at-.bulheller.com
%
% Website:         www.bulheller.com
%
% Mail address:    webmaster.-at-.bulheller.com
%
% Version:         $Revision: 4630 $, $Date: 2009-07-11 08:16:26 +0100 (Sat, 11 Jul 2009) $
%
% Licence:         This program is free software: you can redistribute it and/or modify
%                  it under the terms of the GNU General Public License as published by
%                  the Free Software Foundation, either version 3 of the License, or
%                  (at your option) any later version.
%
%                  This program is distributed in the hope that it will be useful,
%                  but WITHOUT ANY WARRANTY; without even the implied warranty of
%                  MERCHANTABILITY or FITNESS FOR A PARTICULAR PURPOSE.  See the
%                  GNU General Public License for more details.
%
%                  You should have received a copy of the GNU General Public License
%                  along with this program.  If not, see http://www.gnu.org/licenses/.
%
% #####################################################################################################################

\usepackage{graphics}
\usepackage[super, square, sort&compress]{natbib}
\usepackage[margin=2.5cm]{geometry}

\setlength{\parindent}{0em}
\setlength{\parskip}{0.5em}

\title{GetParameters.pm}
\author{Benjamin Bulheller\\ \small www.bulheller.com\\ \small webmaster.-at-.bulheller.com}

\begin{document}
\maketitle

\tableofcontents

\newpage

% ==================================================================================================

\section{Introduction}

\verb'GetParameters.pm' is a Perl library for parsing command line options given to a script. There are quite a few packages like this out there (\verb'GetOpt', \verb'GetOpt::EvaP', \verb'GetOptions') and some of them may be more powerful than this. However, this package was created over three years while working on a PhD thesis and at that time a small package was needed which could quickly be extended about needed features and it was easier to write a new package.

For example, one thing that was needed was a single hash structure containing all given parameters instead of assigning the parameters to individual variables (\verb'GetParameters' creates the hash \verb'$Options' containing the parsed command line). This particularly comes in handy when all parameters have to be passed on to a subroutine, avoiding lots of global variables.

A desired type definition (apart from e.g.\ integer or string) was ``file'', which would automatically check for existence of the given file(s). This includes a check for multiple extensions if only the base name was given.

Option bundling (using \texttt{-abc} instead of \texttt{-a -b -c}) was not implemented, with the advantage of not needing double dashes \verb'--' for long option names.

The main routine of this package reads in the command line parameters and returns a hash with the options as keys and the arguments as items. The options can be defined as:

\begin{itemize}
\item switches (true if given);
\item strings, file names, integers or real numbers;
\item lists (read until the next option is reached,  a minimum and maximum number of elements in the list can be defined as well as the type of the elements).
\end{itemize}

The parameters can be defined as optional or mandatory. Lists can be given in batches, that is for example the items for the list \verb'out' are collected:
\begin{verbatim}
scriptname -out file1 file2 file3 -a -b -c -out file4
\end{verbatim}

The predefined option \verb'-h' displays help screen, if given (unless \verb'-h' is defined as one of the program's parameters).  Wild cards (\verb'*.pdb', \verb'chain?.pdb') are expanded automatically and the list of files is returned.


% ==================================================================================================

\newpage

\section{Implementation}

To use the routine, the references to three variables have to be provided: a hash containing the definitions of the parameters, a hash for the parsed options, and a string with the help message. Optionally it is possible to define default values for the options. It is easier to define these three variables as references in the first place:


\begin{verbatim}
#!/usr/bin/perl -w

use GetParameters;

my $Parameters = {      # create a reference to a hash
   f => "string",
};

my $Options    = {      # create a reference to a hash
   f => "example.txt",
};

my $Help       = "\n" . # a string holding the usage information
   "Usage:   \n" .
   "\n";

\end{verbatim}

This, for example, defines the parameter \verb'-f' which can contain any value, read in as a string. By default it contains the string \verb'example.txt'. The following parameter types are defined:

\begin{itemize}
\item \verb'switch':  this returns true if given
\item \verb'string': just any string
\item \verb'integer': an integer number
\item \verb'real': a real number
\item \verb'file': a file (checked for existence)
\item \verb'list': a list of strings by default
\item \verb'stringlist': a list of strings
\item \verb'integerlist': a list of integers
\item \verb'reallist': a list of real numbers
\item \verb'filelist': a list of (existing) files
\end{itemize}

\verb'$Parameters' contains the possible parameters and their definition (type). After parsing \verb'$Options' contains the parameters and their values. It can be used to define default values for the parameters which will be overwritten if the parameter is given by the user. \verb'$Help' is a single string which contains the usage information, displayed if an input error is detected (an undefined option for example) or not parameter is given at all:

\begin{verbatim}
my $Help = "\nHelp string".
   "Displayed when -h is found\n".
   "or no command line parameters are given\n";
\end{verbatim}

The command to parse the command line parameters is

\begin{verbatim}
GetParameters ($Parameters, $Options, $Help)
\end{verbatim}


% ==================================================================================================


\section{Usage Example}

Assuming the following setup

\begin{verbatim}
my $Parameters = {      # create a reference to a hash
   t => "string",
   v => "switch",
   f => "list",
};
\end{verbatim}

``\verb't''' is defined as string, ``\verb'v''' as switch and ``\verb'f''' as list. The following command line given to the script \verb'scriptname':

\begin{verbatim}
scriptname -t somestring -v -f file1 file2 file3
\end{verbatim}

would result into

\begin{verbatim}
$Options->{f} => [file1, file2, file3]
$Options->{t} => "somestring"
$Options->{v} => 1
\end{verbatim}

If the parameter \verb'-h' is found, the help screen is printed. A plain \verb'list' without type specification is a list of strings by default. If it were defined as a \verb'filelist', each given file would be checked for existence and rejected if not found. This also means that a \verb'filelist' should only be used for files that must already exist.

All parameters which cannot be attributed to one of the switches are collected in the array \verb'$Options->{rest}' (see section \ref{Sec:rest}).

% ==================================================================================================

\section{Parameter Types}

\subsection{Switches}

If a parameter is defined as \verb'switch' like this

\begin{verbatim}
$Parameters = {
   s => "switch"
}
\end{verbatim}

it is true, if given, and false if not given by the user. In a script it could be evaluated like this:

\begin{verbatim}
if ($Options->{s}) { .... }
              else { .... }
\end{verbatim}

If for some reason a switch was defined to be true before the actual command line parameters are parsed (e.g.\ because a configuration file was read), it can be forced to be false by adding ``=0'' to the parameter (without a blank!):

\begin{verbatim}
scriptname -v=0
\end{verbatim}


\subsection{Integers and Real Numbers}

The types \verb'integer' and \verb'real' define a number which is checked for being an integer or floating point value.


\subsection{Files}

A \verb'file' parameter defines the name of a file which has to exist. To read the name of a not yet existing output file for example one would simply read a string. In order to additionally check for one or more extensions, these can also be defined in curly braces with a pipe symbol as delimiter:

\begin{itemize}
\item \verb'f => "file"' and then \verb'-f myfile'

Looks for exactly the given string \verb'myfile'

\item \verb'f => "file{txt}"' and then \verb'-f myfile'

looks for \verb'myfile' and \verb'myfile.txt'

\item \verb'f => "file{doc|txt}"' and then \verb'-f myfile'

looks for \verb'myfile', \verb'myfile.doc' and \verb'myfile.txt'
\end{itemize}


% ==================================================================================================

\section{The \texttt{\$Options->\{rest\}} Hash}

\label{Sec:rest}

The command line is parsed one parameter after another and the arguments are handled according to their definition. Options starting with a dash (\texttt{-}) are regarded as parameters, unless the dash is recognized as the minus symbol of a numeric value. All items following a parameter defined as a list will be added to this parameter's array until the next parameter is reached. All other items (not following a list parameter) are collected in \verb'$Options->{rest}'. Consider the following parameter definition

\begin{verbatim}
my $Parameters = {      # create a reference to a hash
   t => "string",
   v => "switch",
   f => "list",
};
\end{verbatim}

and this command line:

\begin{verbatim}
scriptname stuff1 -t MyString stuff2 -v stuff3 stuff4 -f file1 file2 file3
\end{verbatim}

The first item, \verb'stuff1' is not preceded by any parameter and is hence added to \verb'{rest}' right away. \verb'-t' is recognized as a parameter defined as string and hence only one value, \verb'MyString', is read while the one following it goes into \verb'{rest}'. \verb'-v' is a switch and only causes this hash key to be set true. The items following \verb'-f' are all added to it since it is defined as a list. All this would result into this \verb'$Options' hash:

\begin{verbatim}
$Options->{f}    => [file1, file2, file3]
$Options->{t}    => "MyString"
$Options->{v}    => 1
$Options->{rest} => [stuff1, stuff2, stuff3, stuff4]
\end{verbatim}

The \verb'{rest}' ``parameter'' may be given a definition as well. This may be necessary if a script is only given filenames or integer numbers without the need to use a specific option for that. This definition can be e.g. a \verb'filelist', \verb'reallist', \verb'integerlist', etc. but always has to be a list: 

\begin{verbatim}
my $Parameters = {      # create a reference to a hash
  rest => "filelist",
};
\end{verbatim}

By default it is assigned the value \verb'stringlist'. Strings that contain only digits, a leading dash and periods, that is negative numbers, cannot be used as parameters as they would be mistaken for a negative number. In the following command line '-1.2' would therefore be treated as a negative number:

\begin{verbatim}
scriptname stuff -numbers 3.4 -1.2 7.71 -string perl 
\end{verbatim}

with \verb'-numbers' being e.g. a \verb'reallist' and \verb'-string' a simple string would result into

\begin{verbatim}
$Options->{rest}    => [stuff]
$Options->{numbers} => [3.4, -1.2, 7.71]
$Options->{string}  => "perl"
\end{verbatim}



% ==================================================================================================

\section{Define Mandatory Parameters}

An asterisk at the end of the type declaration marks a parameter as mandatory, for example

\begin{verbatim}
$Parameters{t} = "string*";
\end{verbatim}

% ==================================================================================================

\section{Define the Minimum/Maximum Size of a List}

The ``\verb'list''' declaration can be followed by ``\verb'[min,max]''' to define a range for the size of the list or ``\verb'[value]''' to define a required size of it. This also works for \verb'{rest}' to catch a wrong number of parameters in addition to the ones sorted by switches.

Examples:

\begin{verbatim}
$Parameters->{l} = "list[3]";    # exactly 3 elements
$Parameters->{l} = "list[3]*";   # exactly 3 elements, mandatory
$Parameters->{l} = "list[2,4]";  # 2, 3 or 4 elements
$Parameters->{l} = "list[2,]";   # at least 2 elements
$Parameters->{l} = "list[,3]";   # at most 3 elements
\end{verbatim}

This works with all defined lists:
\begin{itemize}
\item \verb'list': a list of strings by default
\item \verb'stringlist': a list of strings
\item \verb'integerlist': a list of integers
\item \verb'reallist': a list of real numbers
\item \verb'filelist': a list of (existing) files
\end{itemize}

A file list may also contain one or more possible extensions in curly braces and it does not matter in which order the extensions or range is given, i.e.\ \verb'filelist[2]{txt}' and \verb'filelist{txt}[2]' are equivalent.

To end the list input and start e.g. with the files to process (which will be stored
in \verb'$Options->{rest}'), use the \verb'--' parameter:

\begin{verbatim}
scriptname -range 150 250 -- *.cd
\end{verbatim}

This is not needed in case another parameter follows.

% ==================================================================================================

\section{Differentiate Between Given Parameters and Default Values}

Parameters given via the command line overwrite default parameters. Sometimes it is necessary to determine whether a parameter was really given via the command line or whether it was just the default value that had been taken. For this purpose the \verb'$Options->{GivenParams}' key is generated, which contains exactly the given and unaltered parameters.

The library itself needs this value if list parameters are split and given in multiple batches to the script:

\begin{verbatim}
prog  -l as er ty  -f file  -l fg cx
\end{verbatim}

would result in a list ``\verb'l''' containing the five given elements. When the first three items are read the entry \verb'l' in \verb'$WasGiven' is false and the library will delete the default values, if any are defined. When the last two items are read, the \verb'$WasGiven' entry is true and the three items already present in the array are recognized as having been given via the command line and are not deleted.

\end{document}
